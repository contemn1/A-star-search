%%%%%%%%%%%%%%%%%%%%%%%%%%%%%%%%%%%%%%%%%
% University/School Laboratory Report
% LaTeX Template
% Version 3.1 (25/3/14)
%
% This template has been downloaded from:
% http://www.LaTeXTemplates.com
%
% Original author:
% Linux and Unix Users Group at Virginia Tech Wiki 
% (https://vtluug.org/wiki/Example_LaTeX_chem_lab_report)
%
% License:
% CC BY-NC-SA 3.0 (http://creativecommons.org/licenses/by-nc-sa/3.0/)
%
%%%%%%%%%%%%%%%%%%%%%%%%%%%%%%%%%%%%%%%%%

%----------------------------------------------------------------------------------------
%	PACKAGES AND DOCUMENT CONFIGURATIONS
%----------------------------------------------------------------------------------------

\documentclass{article}

\usepackage[version=3]{mhchem} % Package for chemical equation typesetting
\usepackage{siunitx} % Provides the \SI{}{} and \si{} command for typesetting SI units
\usepackage{graphicx} % Required for the inclusion of images
\usepackage{amsmath} % Required for some math elements 
\usepackage{caption}
\usepackage{amssymb}
\usepackage{blkarray}
\setlength\parindent{0pt} % Removes all indentation from paragraphs

\renewcommand{\labelenumi}{\alph{enumi}.} % Make numbering in the enumerate environment by letter rather than number (e.g. section 6)

%\usepackage{times} % Uncomment to use the Times New Roman font

%----------------------------------------------------------------------------------------
%	DOCUMENT INFORMATION
%----------------------------------------------------------------------------------------

\title{Assignment 2} % Title

\author{Xunjie \textsc{Zhu}  \& Xuan \textsc{Wang}} % Author name
\date{\today} % Date for the report

\begin{document}

\maketitle % Insert the title, author and date

\begin{center}
\begin{tabular}{l r}
\end{tabular}
\end{center}

% If you wish to include an abstract, uncomment the lines below
% \begin{abstract}
% Abstract text
% \end{abstract}

%----------------------------------------------------------------------------------------
%	SECTION 1
%----------------------------------------------------------------------------------------



%----------------------------------------------------------------------------------------
%	SECTION 1
%----------------------------------------------------------------------------------------

\section{Problem 1}

%----------------------------------------------------------------------------------------
%	SECTION 3
%----------------------------------------------------------------------------------------

\section{Problem 2}
The most constrained variable is the variable that is most likely to get constrained to no possibilities and force backtracking, so that it can prune the search tree. Least-constraining-value heuristic is trying to leave the maximum flexibility for subsequent variable assignment. Since we only need to find one solution, it's beneficial to look for most likely values first.
%----------------------------------------------------------------------------------------
%	SECTION 4
%----------------------------------------------------------------------------------------

\section{Problem 4}
\subsection{Question a}
let P(X) be the probability that variable X be true, and P($\neg$X) be the probability that variable X be false
\begin{align*}
P(A, B, C, D, E) &= P(A)P(B)P(C)P(D| A, B)P(E| B, C)\\
  &= 0.2 * 0.5 * 0.8 * 0.1 *0.3\\
  &= 0.0024
\end{align*}

\subsection{Question b}
\begin{align*}
P(\neg A, \neg B, \neg C, \neg D, \neg E) &= P(\neg A)P(\neg B)P(\neg C)P(\neg D| \neg A, \neg B) P(\neg E|\neg B, \neg C)\\
  &= 0.8 * 0.5 * 0.2 * 0.9 *0.2\\
  &= 0.0144
\end{align*}

\subsection{Question c}
\begin{align*}
P(\neg A| B, C, D, E) &=  \alpha P(\neg A)P(B)P(C)P(D| A, B)P(E|B, C)\\
  &= \alpha  0.8 * 0.5 * 0.8 * 0.6 *0.3\\
  &= 0.96
\end{align*}


%----------------------------------------------------------------------------------------
%	SECTION 5
%----------------------------------------------------------------------------------------

\section{Problem 6}
\subsection{Question a}
The variable is independent of its non-descendants given its parents. So
\begin{align*}
P(X | MB(X)) &= P(X | Parents(X), Children(X)) \\
	&= \alpha P(X, Parents(X), Children(X)) \\
	&= \alpha P(X, Parents(X), Y)) \\
	&= \alpha P(X| Parents(X)) \prod_{Y_i} P(Y_i | Parents(Y_i))
\end{align*}

\subsection{Question b}
There two non evidence variable Cloudy and Rain and each variable has 2 possible states \{true, false\}, so  there are 4 possible states for the approach to consider.

\subsection{Question c} 
There are 4 possible initial states for variables  Cloudy, Sprinkler, Rain and Wet Grass, \{true, true, false, true\},  \{true, true, true, true\},  \{false, true, true, true\}, \{false, true, false, true\}
\begin{itemize}
\item When the initial state is \{true, true, false, true\}, the first non evidence variable can be Cloudy or Rain
\begin{itemize}
\item When Cloudy is sampled first, P(Cloudy| Sprinkler=true, Rain = false)=0.04. Cloudy is false. The current new state is  \{false, true, false, true\}.  
\item Then Rain should be sampled. P(Rain=True| Sprinkler=True, Cloudy=False, Wet Grass=True) = 0.22. Rain is false. The current new state is  \{false, true, false, true\}.
\item When the rain is sampled first, P(Rain=True| Sprinkler=True, Cloudy=True, Wet Grass=True) = 0.81, The current new state is  \{true, true, true, true\}. 
\item Then Cloudy should be sampled,  P(Cloudy| Sprinkler=True, Rain = True)=0.44. Cloudy is false. The current new state is  \{false, true, true, true\}.
\end{itemize}
\item When the initial state is \{true, true, true, true\}, the first non evidence variable can be Cloudy or Rain
\begin{itemize}
\item When Cloudy is sampled first, P(Cloudy| Sprinkler=True, Rain = True)=0.44. Cloudy is false. The current new state is  \{false, true, true, true\}.  
\item Then Rain should be sampled. P(Rain=True| Sprinkler=True, Cloudy=False, Wet Grass=True) =0.22. Rain is false. The current new state is  \{false, true, false, true\}.
\item When the rain is sampled first, P(Rain=True| Sprinkler=True, Cloudy=True, Wet Grass=True) = 0.81, The current new state is  \{true, true, true, true\}. 
\item Then Cloudy should be sampled,  P(Cloudy| Sprinkler=True, Rain = True)=0.44. Cloudy is false. The current new state is  \{false, true, true, true\}.
\end{itemize}
\item When the initial state is \{false, true, true, true\}, the first non evidence variable can be Cloudy or Rain
\begin{itemize}
\item When Cloudy is sampled first, P(Cloudy| Sprinkler=true, Rain = true)=0.44. Cloudy is false. The current new state is  \{false, true, true, true\}.  
\item Then Rain should be sampled. P(Rain=True| Sprinkler=true, Cloudy=false, Wet Grass=True) =0.22. Rain is false. The current new state is  \{false, true, false, true\}.
\item When the rain is sampled first, P(Rain=True| Sprinkler=true, Cloudy=false, Wet Grass=True) = 0.22, The current new state is  \{false, true, false, true\}. 
\item Then Cloudy should be sampled,  P(Cloudy| Sprinkler=true, Rain = false)=0.04. Cloudy is false. The current new state is  \{false, true, false, true\}.
\end{itemize}
\item When the initial state is \{false, true, false, true\}, the first non evidence variable can be Cloudy or Rain
\begin{itemize}
\item When Cloudy is sampled first, P(Cloudy| Sprinkler=true, Rain = false)=0.04. Cloudy is false. The current new state is  \{false, true, false, true\}.  
\item Then Rain should be sampled. P(Rain=True| Sprinkler=true, Cloudy=false, Wet Grass=True) =0.22. Rain is false. The current new state is  \{false, true, false, true\}.
\item When the rain is sampled first, P(Rain=True| Sprinkler=true, Cloudy=false, Wet Grass=True) = 0.22, The current new state is  \{false, true, false, true\}. 
\item Then Cloudy should be sampled,  P(Cloudy| Sprinkler=true, Rain = false)=0.04. Cloudy is false. The current new state is  \{false, true, false, true\}.
\end{itemize}
\end{itemize}
According to all the circumstances above, the transition matrix should be 

\[
\begin{blockarray}{cccccc}
\{t, t, t, t\} & \{t, t, f, t\} & \{f, t, t, t\} & \{f, t, f, t\} \\
\begin{block}{(ccccc)c}
  0.33 & 0 & 0.67 & 1 & \{t, t, f, t\} \\
  0.5 & 0 & 0 & 0.5 &  \{t, t, t, t\} \\
  0 & 0 & 0.33 & 0.67 & \{f, t, t, t\} \\
  0 & 0 & 0 & 1 & \{f, t, f, t\} \\
\end{block}
\end{blockarray}
 \]




%----------------------------------------------------------------------------------------
%	SECTION 6
%----------------------------------------------------------------------------------------


%----------------------------------------------------------------------------------------
%	BIBLIOGRAPHY
%----------------------------------------------------------------------------------------

\bibliographystyle{IEEEtran}

\bibliography{sample}

%----------------------------------------------------------------------------------------


\end{document}